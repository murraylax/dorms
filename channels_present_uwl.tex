\documentclass{beamer}
\usepackage{beamerthemeshadow}
\usepackage{verbatim}

\usepackage{lastpage}
\usepackage{xcolor}
\usepackage{pgf}
\usepackage{colortbl}
\usepackage{hyperref}

\newcommand{\bi}{\begin{itemize}}
\newcommand{\ei}{\end{itemize}}
\newcommand{\be}{\begin{enumerate}}
\newcommand{\ee}{\end{enumerate}}
\newcommand{\bd}{\begin{description}}
\newcommand{\ed}{\end{description}}
\newcommand{\prbf}[1]{\textbf{#1}}
\newcommand{\prit}[1]{\textit{#1}}
\newcommand{\beq}{\begin{equation}}
\newcommand{\eeq}{\end{equation}}
\newcommand{\bdm}{\begin{displaymath}}
\newcommand{\edm}{\end{displaymath}}

\newcommand{\ft}[1]{
  \frametitle{\begin{tabular}{p{4.2in}r} \textcolor{white}{#1} & \small{\insertframenumber / \inserttotalframenumber} \end{tabular}}
  \setbeamercovered{transparent=18}
}


\newcommand{\stepinv}{\setbeamercovered{invisible}}
\newcommand{\stopinv}{\setbeamercovered{transparent=18}}
\newcommand{\uncoverinv}[1]
{
  \setbeamercovered{invisible}
  \uncover<+->{#1}
  \setbeamercovered{transparent=18}
}
\newcommand{\ans}[1]{\textcolor{blue}{#1}}
\newcommand{\ansinv}[1]
{
  \setbeamercovered{invisible}
  \uncover<+->{\textcolor{blue}{#1}}
  \setbeamercovered{transparent=18}
}
\newcommand{\setinv}{\setbeamercovered{invisible}}
\newcommand{\setvis}{\setbeamercovered{transparent=18}}
\newcommand{\centerpic}[2]
{
  \begin{center}
  \includegraphics[#1]{#2}
  \end{center}
}
\newcommand{\h}[1]{\hat{#1}}
\newcommand{\ds}{\displaystyle}

%\definecolor{light}{rgb}{1.0,0.33,0.33}
\definecolor{light}{rgb}{1.0,0.5,0.5}
\newcommand{\hl}[1]{\alt<#1>{\rowcolor{light}\hspace*{-2.1pt}} {\hspace*{-2.1pt}} }

\definecolor{mycolor}{rgb}{0.6,0.0,0.0}
\usecolortheme[named=mycolor]{structure}

\title[Academic Benefits of Living On Campus]{Academic Benefits of Living On Campus}
\author[James Murray, Department of Economics]{
Pedro de Araujo\inst{1} \and James Murray\inst{2}}
\institute{
\inst{1}
Department of Economics and Business\newline 
Colorado College 
\and
\inst{2}
Department of Economics\newline 
University of Wisconsin - La Crosse 
}
\date{January 21, 2010}

\begin{document}

\frame{\titlepage \setcounter{framenumber}{0}}

\section{Introduction}
\subsection{Academic Benefits to Living on Campus}
\frame
{
  \ft{Measuring Impact on Grade Point Average}
  \uncover<+->{
    Measure the impact of living on campus on students' academic performance, both immediately and in the long-run.
  }

  \uncover<+->{
    \begin{block}{Dependent Variables}
       \bi
       \item Single semester GPA: used to measure immediate effects.
       \item Cumulative GPA: used to measure permanent effects.
       \ei
    \end{block}
  }

  \uncover<+->{
    \begin{block}{Explanatory (Treatment) Variables}
      \bi
      \item Student lived on-campus during Spring 2008: used to measure immediate effects.
      \item Student lived on-campus during any time in the past: used to measure permanent effects.
      \ei
    \end{block}
  }
}

\frame[shrink]
{
  \ft{Measuring Channels for Improved Performance}
  \uncover<+->{
    \begin{block}{Campus Resources}
    Are students that live on campus...
    \bi
    \item study with campus resources: common study areas, computer labs, libraries?
    \item more likely to see a tutor?
    \item more likely to engage in extra-curricular activities?
    \item more likely to use university-provided fitness centers?
    \item spend more time studying?
    \ei
    \end{block}
  }

  \uncover<+->{
    \begin{block}{Peer Influences}
    Are students that live on campus...
    \bi
    \item more likely to study with roommates and/or classmates?
    \item less likely to engage in drugs and alcohol with peers?
    \ei
    \end{block}
  }
}

\subsection{Identifying Implications for Policy}
\frame
{
  \ft{Policy Significance}
  \uncover<+->{
    \begin{block}{Policy Questions}
      \be
      \item Can changing residence hall resources and/or residence hall policies effect academic performance?
      \item If so, \textit{how?}
      \ee	
    \end{block}
  }
  \uncover<+-> {
  \begin{block}{Search for Causation}
    \bi
    \item Essential to establish \textit{causation} for policy implications.
    \item A laboratory scientist would \textit{randomly (independently)} assign subjects to a control and treatment group.
    \item Instrumental Variable Regression: statistical technique that identifies an \textit{independent variable} to identify causation.
    \ei
  \end{block}
  }
}

\begin{frame}
  \ft{Instrumental Variable Regression}
  \uncover<+->{
  \begin{block}{Sample Selection Bias}
    \bi
    \item Subjects \textit{are not randomly} put into treatment and control groups.
    \item More highly motivated students \textit{may choose} to live in dorms.
    \item Students who know they could use the benefits from living on campus \textit{may choose} to live in dorms.
    \ei
  \end{block}}

  \uncover<+->{
  \begin{block}{Instrumental Variables}
    \bi
    \item Find variable(s) \textit{unrelated to academic performance} that influence treatment/control assignment.
    \item Instruments: distance of hometown from school, denied housing due to space limitations.
    \ei
  \end{block}}

\end{frame}


\subsection{Literature}
\frame
{
  \ft{Literature}
  \uncover<+-> {
    \begin{block}{On Campus Residence}
      \bi
      \item Positive impact for freshman: Thompson, et. al. (1993).
      \item No difference: Delucchi (1993).
      \item Critical thinking skills: Pascarella et. al. (1993): 
      \item Social development skills: Flowers (2004).
      \item Positive impact for first-gen students: Pike and Kuh (2005).
      \ei
    \end{block}
  }
}

\frame[shrink]
{
  \ft{Literature}

  \uncover<+->{
    \begin{block}{Peer Influences}
      \bi
      \item Positive influences are dominant: Henderson et. al. (1978).
      \item Negative influences carry through college: Betts and Morell (1999).
      \item ``Average'' students most susceptible to peer influence: Zimmerman (2003).
      \ei
    \end{block}
  }

  \uncover<+->{
    \begin{block}{Campus Resources}
      \bi
      \item Faculty student interaction: Pascarella and Terenzini (1991), Astin (1993), Kuh and Hu (2001a)
      \item Information technology: Kuh and Hu (2001b)
      \item Institutional spending / not necessarily academic support: Toutkoushian and Smart (2001)
      \ei
    \end{block}
  }
}

\frame
{
  \ft{Need for More Research}
  \bi
  \item Find evidence of causation.
  \item Investigate the \textit{channels} of dormitory influences.
  \item Changes in student characteristics and features of higher learning likely changes how students learn: Pascarella and Terenzini (1991).
  \ei
}

\section{Data}
\subsection{Population and Sample}
\frame[shrink]
{
  \ft{Population and Sample}

  \uncover<+->{
  \begin{block}{Population}
  \bi
  \item Undergraduate students at Indiana University Purdue University - Indianapolis.
  \item Approximately 19,700 students under age 25.
  \item Extremely limited on-campus housing capacity: 1,107.
  \item No on-campus housing requirements.
  \ei
  \end{block}
  }

  \uncover<+->{
  \begin{block}{Sample}
  \bi
  \item Electronic survey given to 6,000 undergraduate in Fall 2008.
  \item 363 completed questionnaire [see Sax et. al. (2003)]
  \item Questions included: living situation, social life, study habits, campus resource utilization, cultural background, academic background.
  \ei
  \end{block}
  }
}

\subsection{Dependent and Explanatory Variables}
\frame[shrink]
{
  \ft{Dependent and Explanatory Variables}
  \uncover<+->{
  \begin{block}{Measure of academic performance}
    \bi
    \item Semester GPA.
    \item Cumulative GPA.
    \ei
    ~~~(Each examined in turn)
  \end{block}
  }

  \uncover<+->{
  \begin{block}{Living on campus dummy}
    \bi
    \item Student lived on campus in concurrent semester.
    \item Student lived on campus anytime while at IUPUI.
    \item Student lived on campus in any prior semester.
    \ei
  \end{block}
  }
}

\subsection{Instruments and Controls}
\frame[shrink]{
  \ft{Instruments and Controls}
  \uncover<+->{
  \begin{block}{Instrumental variables}
   \bi
   \item Distance of hometown from campus - positively related to whether a student lived on-campus.
   \item On-campus housing turned down due to lack of available space (dummy).
   \ei
  \end{block}
  }

  \uncover<+->{
  \begin{block}{Controls}
    \bi
    \item Gender
    \item Parents' income
    \item Non-traditional student dummy (age$>$25)
    \item ACT/SAT percentiles
    \item Number of semesters completed
    \item Number of credits in Spring 2008.
    \ei
  \end{block}
  }
}

\subsection{Channel Variables}
\frame[shrink]
{
  \ft{Channel Variables}
  \uncover<+->{
  \begin{block}{University Provided Resources: Fall 2008}
    \bi
    \item Use of fitness resources (hours per week -- Tobit).
    \item Use of tutors (hours per week - Robust OLS).
    \item Engagement in extra-curricular activities (dummy - Probit).
    \item Hours using campus resources (hours per week - Tobit).
    \item Hours studying (hours per week - Tobit).
    \ei
  \end{block}
  }

  \uncover<+->{
  \begin{block}{Peer-Influenced Variables: Fall 2008}
    \bi
    \item Number of drinks per week (Robust OLS)
    \item Ever used drugs while at IUPUI (Probit)
    \item Study with roommates (hours per week - Tobit)
    \item Study with classmates (hours per week - Tobit)
    \ei
  \end{block}
  }
}

\section{Estimation}
\subsection{Academic Benefits}
\frame
{
  \ft{Estimating Academic Benefits}
  \uncover<+->{
  \begin{block}{Estimation Procedure}
    \be
    \item OLS (No instruments / no control for self-selection bias)
    \item IV: Just-identified using only distance from campus.
    \item GMM using both instruments.
    \item Two-stage MLE (first stage probit) using both instruments.
    \ee
  \end{block}
  }

  \uncover<+->{
  \begin{block}{Three Specifications}
    \be
    \item Cumulative GPA on DORM\_EVER.
    \item Spring Semester 2008 GPA on DORM\_EVER.
    \item Spring Semester 2008 GPA on DORM\_S08.
    \ee
  \end{block}
  }
}

\frame
{
  \ft{Results for Academic Benefits}
  \footnotesize{
  \begin{center}
  \begin{tabular}{cccc}
    \multicolumn{4}{c}{\textbf{Coefficient on Living on Campus Dummy}}\\\\\hline\hline
    \multicolumn{4}{c}{Cumulative GPA on DORM\_EVER} \\ \hline
    ~~~~~~OLS~~~~~~ & ~~~~~~IV~~~~~~ & ~~~~~~GMM~~~~~~ & ~~~~~~MLE~~~~~~ \\ \hline
    0.210** & 0.312* & 0.448*** & 0.431***\\
    $[0.087]$ & [0.187] & [0.140] & [0.156] \\ \hline \hline
    \multicolumn{4}{c}{Spring 2008 Semester GPA on DORM\_EVER} \\ \hline
    OLS & IV & GMM & MLE \\ \hline
    0.185* & 0.221 & 0.416** & 0.410**\\
    $[0.095]$ & [0.289] & [0.212] & [0.166]\\ \hline\hline
    \multicolumn{4}{c}{Spring 2008 Semester GPA on DORM\_S08} \\ \hline
    OLS & IV & GMM & MLE \\ \hline
    0.303*** & 0.490 & 0.973* & 0.693*** \\
    $[0.096]$ & [0.642] & [0.526] & [0.201] \\\hline\hline
    \multicolumn{4}{l}{\scriptsize{Standard errors in parenthesis.}}
  \end{tabular}
  \end{center}
  }
}

\subsection{Channels}
\frame{
  \ft{Estimating Channels}
  \bi
  \item Explanatory Variables:
    \bi
    \item DORM\_PAST: Whether or not student lived on campus in the past.
    \item DORM\_F08: Whether or not student lived on campus in Fall 2008 semester.\newline (Both included simultaneously)
    \item Same set of controls.
    \ei
  \item No IV estimation:
    \bi
    \item Computationally, it's hard with limited dependent variables.
    \item Limited sample size and limited explanatory power.
    \ei
  \ei
}

\frame{
  \ft{Results for Campus Resources}
  \footnotesize{
    \begin{center}
      \begin{tabular}{l|ccccc} 
        \multicolumn{6}{c}{\textbf{Campus Resource Variables}} \vspace*{0.4pc}\\ \hline\hline
         & FITNESS & TUTORS & XTCUR & CAMPUS & STUDY  \\ 
         & Tobit & Robust OLS & Probit & Tobit & Tobit \\ \hline
        DORM\_F08	&	-3.687**	&	0.153	&	0.788*	&	-6.613***	&	-1.702\\
	&	[1.459]	&	[0.136]	&	[0.429]	&	[2.066]	&	[1.55]	\\ \hline
        DORM\_PAST	&	0.023	&	-0.279**	&	0.937***	&	0.916	&	1.296\\
	&	[1.069]	&	[0.11]	&	[0.268]	&	[1.532]	&	[1.317]	\\ \hline
        N & 207 & 225 & 232 & 231 & 225 \\
        F-stat & 1.67 & 1.46 & --- & 3.09*** & 1.46  \\
        Wald Stat & --- & --- & 50.45*** & --- & ---  \\
        (Pseudo) $R^2$ & 0.0163 & 0.0206 & 0.1663 & 0.0228 & 0.0025 \\ \hline\hline
      \end{tabular}
  \end{center}
    \bi
    \item Except for extra-curricular activities, significant values have opposite than expected signs.
    \item Engaging in extra-curricular activities has an immediate and permanent effect.
    \ei
  }
}

\frame{
  \ft{Results for Peer Influences}
  \footnotesize{
    \begin{center}
      \begin{tabular}{l|cccc}
        \multicolumn{5}{c}{\textbf{Peer-Influenced Variables}} \vspace*{0.4pc} \\ \hline\hline
        & DRINKS & DRUGS & STUDCLASS & STUDROOM  \\ 
        & Robust OLS & Probit & Tobit & Tobit \\ \hline
        DORM\_F08	&	-0.186	&	0.200	&	0.051	&	2.077	\\
	&	[0.183]	&	[0.389]	&	[1.156]	&	[1.803]	\\ 
        DORM\_PAST	&	-0.341***	&	0.204	&	2.313***	&	2.467**	\\	
        &	[0.131]	&	[0.312]	&	[0.812]	&	[1.218]	\\ \hline
        N & 226 & 230 & 231 & 230 \\
        F-stat & 4.58*** & --- & 2.37** & 3.50***  \\
        Wald Stat & --- & 26.98*** & --- & --- \\
        (Pseudo) $R^2$ & 0.1322 & 0.1140 & 0.0272 & 0.0601  \\ \hline \hline
      \end{tabular}
  \end{center}
  }
  Delayed but significant long term effects:
  \bi
  \item Less likely to consume alcohol.
  \item More likely to study with peers. 
  \ei
}

\section{}
\subsection{Conclusion}
\frame
{
  \ft{Conclusion}
  \uncover<+->{
    \begin{block}{Academic Benefits}
      \bi
      \item Immediate effect: estimates range from 0.303 (OLS) to 0.973 (IV/GMM) increase in semester GPA.
      \item Permanent effect: estimates range from 0.210 (OLS) to 0.448 (IV/GMM) increase in cumulative GPA.
      \ei
    \end{block}
  }
  \uncover<+->{
    \begin{block}{Channels}  
      \bi
      \item More likely to develop productive relationships with peers.
      \item Consume less alcohol in \textit{subsequent} semesters.
      \item More likely to participate in extra-curricular activities, stay involved.
      \item Largely failed to identify channels to explain an immediate effect.
      \ei
    \end{block}
  }
}






\end{document}

