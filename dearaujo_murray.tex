\documentclass[12pt]{article}
\usepackage[T1]{fontenc}
\usepackage{calc}
\usepackage{setspace}
\usepackage{multicol}
\usepackage{fancyheadings}

\usepackage{graphicx}
\usepackage{color}
\usepackage{rotating}
\usepackage{harvard}
\usepackage{aer}
\usepackage{aertt}
\usepackage{verbatim}

\setlength{\voffset}{0in}
\setlength{\topmargin}{0pt}
\setlength{\hoffset}{0pt}
\setlength{\oddsidemargin}{0pt}
\setlength{\headheight}{0pt}
\setlength{\headsep}{.4in}
\setlength{\marginparsep}{0pt}
\setlength{\marginparwidth}{0pt}
\setlength{\marginparpush}{0pt}
\setlength{\footskip}{.1in}
\setlength{\textwidth}{6.5in}
\setlength{\textheight}{9in}
\setlength{\parskip}{0pc}

\renewcommand{\baselinestretch}{1.1}

\newcommand{\bi}{\begin{itemize}}
\newcommand{\ei}{\end{itemize}}
\newcommand{\be}{\begin{enumerate}}
\newcommand{\ee}{\end{enumerate}}
\newcommand{\bd}{\begin{description}}
\newcommand{\ed}{\end{description}}
\newcommand{\prbf}[1]{\textbf{#1}}
\newcommand{\prit}[1]{\textit{#1}}
\newcommand{\beq}{\begin{equation}}
\newcommand{\eeq}{\end{equation}}
\newcommand{\bdm}{\begin{displaymath}}
\newcommand{\edm}{\end{displaymath}}
\newcommand{\script}[1]{\begin{cal}#1\end{cal}}
\newcommand{\citee}[1]{\citename{#1} (\citeyear{#1})}
\newcommand{\h}[1]{\hat{#1}}
\newcommand{\ds}{\displaystyle}

\newcommand{\app}
{
\appendix
}

\newcommand{\appsection}[1]
{
\let\oldthesection\thesection
\renewcommand{\thesection}{Appendix \oldthesection}
\section{#1}\let\thesection\oldthesection
\renewcommand{\theequation}{\thesection\arabic{equation}}
\setcounter{equation}{0}
}

\pagestyle{fancyplain}
\lhead{}
\chead{Effects of Dormitory Living on Student Performance}
\rhead{\thepage}
\lfoot{}
\cfoot{}
\rfoot{}

\begin{document}

\begin{titlepage}
\begin{singlespace}
\title{Estimating the Effects of Dormitory Living on Student Performance}
\date{\today}
\author{
}

\maketitle

\thispagestyle{empty}

\abstract{A survey will be administered to college students at a large state school to determine what impact dormitory living has on student performance.  Many large universities require freshman to live in dormitories on the basis that living on campus leads to better classroom performance and lower drop out incidence.  Large universities also provide a number of academic services in dormitories such as tutoring and academic student organizations.  While there has been a large literature examining the effects of peer interaction on student achievement, which may play a large role in dormitory living, there has been little empirical work on the predicted impact of living in a dorm instead of private housing off campus.  A survey will be administered at a university that does not require students to live in dorms that asks students to report current grade point average, class attendance rates, use of on campus academic resources, reasons for their housing decision, and other academic and demographic characteristics.  With such a data set we can determine whether living in a dormitory leads to improved student performance and be able to identify the particular channels for these improvements.} \newline 

\noindent \textit{Keywords}: Student performance, dormitory, cross-section analysis, regression, instrumental variables. \\
\noindent \textit{JEL classification}: C13, C21, I21.
\end{singlespace}
\end{titlepage}


\end{document}



