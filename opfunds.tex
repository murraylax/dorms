\documentclass[12pt]{article}
\usepackage[T1]{fontenc}
\usepackage{calc}
\usepackage{setspace}
\usepackage{multicol}
\usepackage{fancyheadings}

\usepackage{graphicx}
\usepackage{color}
\usepackage{rotating}
\usepackage{harvard}
\usepackage{aer}
\usepackage{aertt}
\usepackage{verbatim}

\setlength{\voffset}{0in}
\setlength{\topmargin}{0pt}
\setlength{\hoffset}{0pt}
\setlength{\oddsidemargin}{0pt}
\setlength{\headheight}{0pt}
\setlength{\headsep}{.1in}
\setlength{\marginparsep}{0pt}
\setlength{\marginparwidth}{0pt}
\setlength{\marginparpush}{0pt}
\setlength{\footskip}{.4in}
\setlength{\textwidth}{6.5in}
\setlength{\textheight}{9in}
\setlength{\parskip}{0pc}

\renewcommand{\baselinestretch}{1.5}

\newcommand{\bi}{\begin{itemize}}
\newcommand{\ei}{\end{itemize}}
\newcommand{\be}{\begin{enumerate}}
\newcommand{\ee}{\end{enumerate}}
\newcommand{\bd}{\begin{description}}
\newcommand{\ed}{\end{description}}
\newcommand{\prbf}[1]{\textbf{#1}}
\newcommand{\prit}[1]{\textit{#1}}
\newcommand{\beq}{\begin{equation}}
\newcommand{\eeq}{\end{equation}}
\newcommand{\bdm}{\begin{displaymath}}
\newcommand{\edm}{\end{displaymath}}
\newcommand{\script}[1]{\begin{cal}#1\end{cal}}
\newcommand{\citee}[1]{\citename{#1} (\citeyear{#1})}
\newcommand{\h}[1]{\hat{#1}}
\newcommand{\ds}{\displaystyle}

\newcommand{\app}
{
\appendix
}

\newcommand{\appsection}[1]
{
\let\oldthesection\thesection
\renewcommand{\thesection}{Appendix \oldthesection}
\section{#1}\let\thesection\oldthesection
\renewcommand{\theequation}{\thesection\arabic{equation}}
\setcounter{equation}{0}
}

\pagestyle{plain}
\begin{document}

\begin{center}\textbf{Opportunity Funds Request to Bring in Co-author / Speaker}\\
James Murray\\
Dahl School of Business\\
Viterbo University\end{center}
\vspace*{2pc}

I am writing to request \$905 (see budget below) from the Opportunity Funds Grant to bring in Dr. Pedro de Araujo, Assistant Professor of Economics at Colorado College, for the period Sunday, February 1 through Saturday, February 7th.  

I am currently co-authoring two papers with Dr. de Araujo (the titles and abstracts for each of these papers is attached) and we are at a point in our research where we need to meet in person for a brief period of time to make progress on the work.

The first paper, ``Estimating the Effects of Dormitory Living on Student Performance,'' examines whether living in dormitories has a positive influence on student performance.  Some casual research has already suggested a linkage, but careful statistical research has not been done to determine a direct causation.  A significant amount of work on this paper has already been completed.  A survey was developed, put online, and tested with a small sample.  Early in the Fall 2008 semester, the survey was finalized and sent to 8000 students at a large public University where living in dormitories is optional.  Data has been collected for 500 students that responded to the survey.  We are at the point to begin analyzing the data and writing the results.  We anticipate submitting this paper for publication by the end of the Spring 2009 semester.

The second paper, ``A Life Insurance Deterrent to the Spread of HIV and AIDS in Africa'', examines an innovated policy prescription to help reduce the spread of HIV and AIDS in impoverished African countries, using theoretical economic modeling and carefully calibrated parameters to match observations.  Dr. de Araujo has done already done a significant amount of literature review on the topic, and the two of us have begun formulating the model.  Building the appropriate model, and solving it, and asking the appropriate policy questions in terms of the model requires further one-on-one collaboration between Dr. de Araujo and myself.  We anticipate submitting this paper for publication during the Fall 2008 semester.

While Dr. de Araujo is here, he has agreed to present to our faculty and students some of his latest empirical research on the relationship between socioeconomic variables and the knowledge and spread of HIV/AIDS in India.  His primary findings are that despite the culture in India being primarily faithful and monogamous, there is still concern especially among the poor, less educated classes.  His empirical results bring up a number of policy recommendations for India that may not work as well in other parts of the world.

Dr. de Araujo has also agreed to visit a class and have lunch with students, although most of his and my time will be needed this week to collaborate on our research projects.

Attached to this document are abstracts of Dr. de Araujo's and my collaborative work, and Dr. de Araujo's curriculum vita, and my curriculum vita.

\begin{center}\textbf{Budget}\end{center}
\vspace*{2pc}
\begin{tabular}{p{5in}l}
\textbf{Item} & \textbf{Amount} \\ \hline
Lunch with DSOB students (15 people at \$7 per plate) & \$105 \\
Dinner with Viterbo Faculty (10 people at \$20 per plate) & \$200 \\
Taxi Service in La Crosse & \$100 \\
Lodging at Professor James Murray's Residence & \$0 \\
Airfare from Colorado Springs, CO to La Crosse, WI & \$500 \\ \hline
\textbf{Total} & \textbf{\$905} \\ \hline
\end{tabular}

\vspace*{1in}

\begin{tabular}{lll}
\vspace*{-1pc}\line(1,0){200} \hspace*{0.5in}  & \line(1,0){100} \hspace*{1in} \\
\scriptsize{Applicant Signature} & \scriptsize{Date} \\ 

\vspace*{-1pc}\line(1,0){200} \hspace*{0.5in}  & \line(1,0){100} \hspace*{1in} \\
\scriptsize{Dean Signature} & \scriptsize{Date} \\

\vspace*{-1pc}\line(1,0){200} \hspace*{0.5in}  & \line(1,0){100} \hspace*{1in} \\
\scriptsize{Academic Vice President Signature} & \scriptsize{Date} \\
\end{tabular}
\newpage

\begin{singlespace}
\begin{center}
\textbf{Estimating the Effects of Dormitory Living on Student Performance}\footnote{PRELIMINARY DRAFT.  The authors are responsible for all errors.}\\
\vspace*{2pc}

Pedro de Araujo\footnote{\textit{Mailing address}: 14 E. Cache La Poudre Street, Colorado Springs, CO  80903\\ \textit{E-mail address}: Pedro.deAraujo@ColoradoCollege.edu.  \textit{Phone number}: (719) 389-6470.}\\Department of Economics and Business\\Colorado College\\
\vspace*{2pc}

James Murray\footnote{\textit{Mailing address}: 900 Viterbo Drive, La Crosse, WI  54601. \textit{E-mail address}: jmmurray@viterbo.edu.  \textit{Phone number}: (608)796-3365.}\\Dahl School of Business\\Viterbo University\\
\vspace*{2pc}
\end{center}

\abstract{A survey will be administered to college students at a large state school to determine what impact dormitory living has on student performance.  Many large universities require freshman to live in dormitories on the basis that living on campus leads to better classroom performance and lower drop out incidence.  Large universities also provide a number of academic services in dormitories such as tutoring and academic student organizations.  While there has been a large literature examining the effects of peer interaction on student achievement, which may play a large role in dormitory living, there has been little empirical work on the predicted impact of living in a dorm instead of private housing off campus.  A survey will be administered at a university that does not require students to live in dorms that asks students to report current grade point average, class attendance rates, use of on campus academic resources, reasons for their housing decision, and other academic and demographic characteristics.  With such a data set we can determine whether living in a dormitory leads to improved student performance and be able to identify the particular channels for these improvements.} \newline 

\noindent \textit{Keywords}: Student performance, dormitory, cross-section analysis, regression, instrumental variables. \\
\noindent \textit{JEL classification}: C13, C21, I21.

\newpage

\begin{center}
\textbf{A Life Insurance Deterrent to the Spread of HIV and AIDS in Africa}\footnote{PRELIMINARY DRAFT.  The authors are responsible for all errors.}\\
\vspace*{2pc}

Pedro de Araujo\footnote{\textit{Mailing address}: 14 E. Cache La Poudre Street, Colorado Springs, CO  80903\\ \textit{E-mail address}: Pedro.deAraujo@ColoradoCollege.edu.  \textit{Phone number}: (719) 389-6470.}\\Department of Economics and Business\\Colorado College\\
\vspace*{2pc}

James Murray\footnote{\textit{Mailing address}: 900 Viterbo Drive, La Crosse, WI  54601. \textit{E-mail address}: jmmurray@viterbo.edu.  \textit{Phone number}: (608)796-3365.}\\Dahl School of Business\\Viterbo University\\
\vspace*{2pc}
\end{center}

\abstract{The spread of HIV and AIDS and risky sexual behavior continues to be a problem in many African countries despite government measures to educate people on the prevalence and severity of the disease and measures to promote safe sex practices such as making condoms readily available at reduced or no cost.  Interview evidence has suggested that people continue to engage in risky sexual behavior because many other conditions exist in these countries that significantly reduce life expectancy.  These include other unpreventable health risks, inaccessibility to health care, and dangerous working conditions such as those in very poor mining regions.  We suggest that using government funds to offer life insurance may be more a effective means of deterring risky sexual behavior than funding safe sex programs.  To evaluate this policy prescription we develop and calibrate an overlapping generations model with bequest motive and endogenous probability of demise.  In the model, agents can receive life insurance benefits if their death is not the result of AIDS.} \newline 

\noindent \textit{Keywords}: HIV, AIDS, life insurance, overlapping generations. \\
\noindent \textit{JEL classification}: H51, I18, I38.
\end{singlespace}

\end{document}



