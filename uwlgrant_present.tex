\documentclass{beamer}
\usepackage{beamerthemeshadow}
\usepackage{verbatim}

\usepackage{lastpage}
\usepackage{xcolor}
\usepackage{pgf}
\usepackage{colortbl}
\usepackage{hyperref}

\newcommand{\bi}{\begin{itemize}}
\newcommand{\ei}{\end{itemize}}
\newcommand{\be}{\begin{enumerate}}
\newcommand{\ee}{\end{enumerate}}
\newcommand{\bd}{\begin{description}}
\newcommand{\ed}{\end{description}}
\newcommand{\prbf}[1]{\textbf{#1}}
\newcommand{\prit}[1]{\textit{#1}}
\newcommand{\beq}{\begin{equation}}
\newcommand{\eeq}{\end{equation}}
\newcommand{\bdm}{\begin{displaymath}}
\newcommand{\edm}{\end{displaymath}}

\newcommand{\ft}[1]{
  %\frametitle{\begin{tabular}{p{4.2in}r} \textcolor{white}{#1} & \small{\insertframenumber / \inserttotalframenumber} \end{tabular}}
  \frametitle{\begin{tabular}{p{4.2in}r} \textcolor{white}{#1} & \small{\insertframenumber / 7} \end{tabular}}
  \setbeamercovered{transparent=18}
}

\newcommand{\eft}[1]{
  \frametitle{\begin{tabular}{p{4in}r} \textcolor{white}{#1} & \small{\hyperlink{f:questions}{\beamergotobutton{GO BACK}}} \end{tabular}}
  \setbeamercovered{transparent=18}
}

\newcommand{\stepinv}{\setbeamercovered{invisible}}
\newcommand{\stopinv}{\setbeamercovered{transparent=18}}
\newcommand{\uncoverinv}[1]
{
  \setbeamercovered{invisible}
  \uncover<+->{#1}
  \setbeamercovered{transparent=18}
}
\newcommand{\ans}[1]{\textcolor{blue}{#1}}
\newcommand{\ansinv}[1]
{
  \setbeamercovered{invisible}
  \uncover<+->{\textcolor{blue}{#1}}
  \setbeamercovered{transparent=18}
}
\newcommand{\setinv}{\setbeamercovered{invisible}}
\newcommand{\setvis}{\setbeamercovered{transparent=18}}
\newcommand{\centerpic}[2]
{
  \begin{center}
  \includegraphics[#1]{#2}
  \end{center}
}
\newcommand{\h}[1]{\hat{#1}}
\newcommand{\ds}{\displaystyle}

%\definecolor{light}{rgb}{1.0,0.33,0.33}
\definecolor{light}{rgb}{1.0,0.5,0.5}
\newcommand{\hl}[1]{\alt<#1>{\rowcolor{light}\hspace*{-2.1pt}} {\hspace*{-2.1pt}} }

\definecolor{mycolor}{rgb}{0.6,0.0,0.0}
\usecolortheme[named=mycolor]{structure}

\title[Academic Benefits of Living On Campus]{Academic Benefits of Living On Campus:}
\subtitle{A Look At Peer Influences and Utilization of University Provided Resources}
\author[James Murray, Department of Economics]{James Murray\\Department of Economics\\University of Wisconsin - La Crosse}
\date{November 19, 2009}

\begin{document}

\frame{\titlepage \setcounter{framenumber}{0}}

\section{Purpose}
\subsection{Grade Point Average}
\frame
{
  \ft{Purpose}
  \uncover<+->{How does living on campus influence academic performance?}
  \bi
  \item<+-> Quantify the impact of living on campus on students' Grade Point Average (GPA).
  \ei
}
  
\subsection{Campus Resources}
\frame
{
  \ft{Campus Resources}
  \uncover<+->{Are differences in GPA due to greater utilization of resources?}\\
  \vspace*{1pc}
  \uncover<+->{Do on-campus students...}
    \bi
    \item<.-> use university academic resources (libraries, computer technology) more often?
    \item<.-> use university provided exercise facilities more often?
    \item<.-> use tutoring services more often?
    \item<.-> go to professors' office hours more often?
    \ei
}

\subsection{Peer Influences}
\frame
{
  \ft{Peer Influences}
  \uncover<+->{Are differences in GPA due to differences in peer influences?}\\
  \vspace*{1pc}
  \uncover<+->{Are on-campus students...}
    \bi
    \item<.-> more likely to study with roommates and/or classmates?
    \item<.-> more likely to join university-sponsored extracurricular activities?
    \item<.-> less likely to engage in drugs and alcohol?
    \item<.-> less likely to have roommates engaging in drugs and alcohol?
    \ei
}

\section{Significance}
\subsection{Policy Significance}

\frame
{
  \ft{Policy Significance}
  \uncover<+->{
    \begin{block}{Policy Questions}
      \be
      \item<+-> Can changing residence hall resources and/or residence hall policies effect academic performance?
      \item<+-> If so, \textit{how?}
      \ee	
    \end{block}
  }
  \uncover<+-> {
  \begin{block}{Search for Causation}
    \bi
    \item<+-> Instrumental Variable Regression: statistical technique used to determine causation.
    \item<+-> Essential for policy implications.
    \ei
  \end{block}
  }
}

\subsection{Scholarly Significance}
\frame
{
  \ft{Scholarly Significance}
  \uncover<+-> {
    \begin{block}{On Campus Residence}
      \bi
      \item<+-> Positive impact on academics: Thompson, et. al. (1993).
      \item<+-> No difference: Delucchi (1993).
      \item<+-> Critical thinking skills: Pascarella et. al. (1993): 
      \item<+-> Social development skills: Flowers (2004).
      \ei
    \end{block}
  }

  \uncover<+->{
    \begin{block}{Peer Influences}
      \bi
      \item<+-> Positive influences are dominant: Henderson et. al. (1978).
      \item<+-> Negative influences carry through college: Betts and Morell (1999).
      \item<+-> ``Average'' students most susceptible to peer influence: Zimmerman (2003).
      \ei
    \end{block}
  }
}

\frame
{
  \ft{Need for More Research}
  \bi
  \item<+-> Find evidence of causation.
  \item<+-> Investigate the \textit{channels} of dormitory influences.
  \item<+-> Changes in student characteristics and features of higher learning likely changes how students learn: Pascarella and Terenzini (1991).
  \ei
}

\section{Methodology}
\frame
{
  \ft{Methodology}
  \uncover<+->{
  \begin{block}{Stage 1: Instrumental Variable Regressions}
    \bi
    \item<+-> Explanatory Variables: Dormitory + Background characteristics (Table 1).
    \item<+-> Dependent Variables: Resource Utilization and Peer Influences (Table 2).
    \ei
  \end{block}
  }

  \uncover<+->{
  \begin{block}{Stage 2: Regression}
    \bi
    \item<+-> Explanatory Variables: Resource Utilization and Peer Influences 
    \item<+-> Dependent Variable: Academic Performance (GPA).
    \ei
  \end{block}
  }
  \uncover<+->{Traces (and quantifies) the \textit{channel} in which living on campus influences academic performance.}

}

\section{}
\subsection{Questions?}
\begin{frame}\label{f:questions}
  \begin{center}Questions?\end{center}
  \begin{center}\hyperlink{f:resources}{\beamergotobutton{Answers}}\end{center}
\end{frame}

\section{}
\subsection{Additional Information}
\begin{frame}\label{f:resources}
  \frametitle{Additional Information}
  \bi
  \item Data \hyperlink{f:data}{\beamergotobutton{GO}}
  \item Table 1: Background control variables \hyperlink{f:background}{\beamergotobutton{GO}}
  \item Table 2: Intermediate variables \hyperlink{f:intermediate}{\beamergotobutton{GO}}
  \item Instrumental Variable Regression \hyperlink{f:iv}{\beamergotobutton{GO}}
  \ei
\end{frame}

\section{}
\subsection{Data}
\begin{frame}\label{f:data}
  \eft{Data}
  \begin{block}{Population}
  \bi
  \item Undergraduate students at Indiana University Purdue University - Indianapolis.
  \item Approximately 19,700 students under age 25.
  \item Extremely limited on-campus housing capacity: 1,107.
  \item No on-campus housing requirements.
  \ei
  \end{block}

  \begin{block}{Sample}
  \bi
  \item Electronic survey given to 6,000 undergraduate in Fall 2008.
  \item 363 completed questionnaire [see Sax et. al. (2003)]
  \item Questions included: living situation, social life, study habits, campus resource utilization, cultural background, academic background.
  \ei
  \end{block}
\end{frame}

\section{}
\subsection{Background Characteristics}
\begin{frame}\label{f:background}
  \eft{Background Characteristics}
  \bi
  \item High School GPA.
  \item ACT/SAT percentile.
  \item Number of college credits earned.
  \item Non-traditional student.
  \item Gender.
  \item Ethnicity.
  \item Parents income.
  \item International student.
  \ei
\end{frame}

\section{}
\subsection{Intermediate Variables}
\begin{frame}\label{f:intermediate}
  \eft{Intermediate Variables}
  \begin{block}{Utilization of University Resources}
  \bi
  \item Hours per week spent studying.
  \item Hours per week using library, campus technology.
  \item Hours per week using campus fitness facilities.
  \item Hours per week attending professors' office hours.
  \item Hours per week using tutors.
  \ei
  \end{block}

  \begin{block}{Peer Influences}
  \bi
  \item Hours per week studying with classmates / roommates.
  \item Involvement in university sponsored extracurricular activities.
  \item Drug and alcohol use.
  \item Peer drug and alcohol use.
  \ei
  \end{block}
\end{frame}

\section{}
\subsection{Instrumental Variable Regression}
\begin{frame}\label{f:iv}
  \eft{Instrumental Variable Regression}
  \uncover<+->{
  \begin{block}{Sample Selection Bias}
    \bi
    \item<+-> Subjects \textit{are not randomly} put into treatment and control groups.
    \item<+-> More highly motivated students \textit{may choose} to live in dorms.
    \item<+-> Comparing on-campus students to off campus students may amount to simply comparing highly motivated students to less motivated students.
    \ei
  \end{block}}

  \uncover<+->{
  \begin{block}{Instrumental Variables}
    \bi
    \item<+-> Find variable(s) \textit{unrelated to academic performance} that help(s) determine treatment/control assignment.
    \item<+-> Instruments: distance of hometown from school, denied housing due to space limitations.
    \ei
  \end{block}}

\end{frame}


\end{document}

